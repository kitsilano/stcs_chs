\documentclass[11pt]{article}
\usepackage[autonum]{macros}

% figures and bibliography
%\usepackage[sorting=nyt,backend=biber,bibstyle=alphabetic,citestyle=alphabetic]{biblatex}
%\usepackage[backend=biber]{biblatex}
\usepackage{natbib}
%\bibliography{ref.bib}


% set margins' lenghts %%%%%%
\setlength{\oddsidemargin}{0.0 in}
\setlength{\evensidemargin}{0.0 in}
\setlength{\topmargin}{-0.6 in}
\setlength{\textwidth}{6.5 in}
\setlength{\textheight}{8.5 in}
\setlength{\headsep}{0.75 in}
\setlength{\parindent}{0 in}
\setlength{\parskip}{0.1 in}

\usepackage{multirow} % for table
\usepackage[table]{xcolor} % alternating row colors in tables
\renewcommand{\abstractname}{Executive Summary} % change title of abstract

% set fancy style for margins %%%%%%
\usepackage{fancyhdr}
\pagestyle{fancy}
\fancyhf{}
\lhead{Madden, C.l}
\chead{UBC STCS}
\rhead{Diluvi, G. C.}
\cfoot{\thepage}





\title{\vspace{-2cm}\Large A Statistical View of The Right to Vancouverism: Social Reproduction Placemaking in the Revanchist City}
\author{\normalsize Gian Carlo Diluvi\footnote{Department of Statistics, University of British Columbia. Contact: \href{mailto:gian.diluvi@stat.ubc.ca}{\texttt{gian.diluvi@stat.ubc.ca}}.}}
\date{\normalsize Client: Cheryl-lee Madden
\vskip 0.1cm
June 2021}




\begin{document}

\maketitle



%\begin{abstract}
%  \noindent Lorem ipsum dolor sit amet, consectetur adipisicing elit, sed do eiusmod tempor incididunt ut labore et dolore magna aliqua. Ut enim ad minim veniam, quis nostrud exercitation ullamco laboris nisi ut aliquip ex ea commodo consequat. Duis aute irure dolor in reprehenderit in voluptate velit esse cillum dolore eu fugiat nulla pariatur. Excepteur sint occaecat cupidatat non proident, sunt in culpa qui officia deserunt mollit anim id est laborum.
%\end{abstract}


\section{Introduction}

reference blog post and talk about the 2018 chs.

statistical questions:

Q1: how sound are statistical analyses in blog post?

Q2: can we expand that analysis to account for low-income women?

Q3: can we deep dive into vancouver?







\section{Statistical Considerations of the 2018 CHS PUMF} \label{sec:stats}

mention that pumf cannot be used for statistical analyses.

give reasons for this, citing the user's guide.

mention that the whole file can be used for this, but access is not public and has to be requested.

also, the pumf is aggregated at the vancouver/other cities/rural level, i.e. there are no neighborhoods in vancouver.

however, the whole file is more granular, so that could be used.

mention possible models that can be used and give references.


\section{Exploratory Data Analysis} \label{sec:eda}

here we replicate some of the plots in the blog post but separating by gender and only for bc.

mention how gender is calculated in survey and limitations.

show plots, explain.




\section{Conclusion} \label{sec:conclusion}





%\clearpage
%\printbibliography
%\bibliographystyle{abbrvnat}
%\bibliography{ref}


\end{document}
