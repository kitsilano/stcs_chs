\documentclass[11pt]{article}
\usepackage[autonum]{macros}

% figures and bibliography
%\usepackage[sorting=nyt,backend=biber,bibstyle=alphabetic,citestyle=alphabetic]{biblatex}
%\usepackage[backend=biber]{biblatex}
\usepackage{natbib}
%\bibliography{ref.bib}


% set margins' lenghts %%%%%%
\setlength{\oddsidemargin}{0.0 in}
\setlength{\evensidemargin}{0.0 in}
\setlength{\topmargin}{-0.6 in}
\setlength{\textwidth}{6.5 in}
\setlength{\textheight}{8.5 in}
\setlength{\headsep}{0.75 in}
\setlength{\parindent}{0 in}
\setlength{\parskip}{0.1 in}

\usepackage{multirow} % for table
\usepackage[table]{xcolor} % alternating row colors in tables
\renewcommand{\abstractname}{Executive Summary} % change title of abstract

% set fancy style for margins %%%%%%
\usepackage{fancyhdr}
\pagestyle{fancy}
\fancyhf{}
\lhead{Madden, C.l}
\chead{UBC STCS}
\rhead{Diluvi, G. C.}
\cfoot{\thepage}





\title{\vspace{-2cm}\Large A Statistical View of The Right to Vancouverism: Social Reproduction Placemaking in the Revanchist City}
\author{\normalsize Gian Carlo Diluvi\footnote{Department of Statistics, University of British Columbia. Contact: \href{mailto:gian.diluvi@stat.ubc.ca}{\texttt{gian.diluvi@stat.ubc.ca}}.}}
\date{\normalsize Client: Cheryl-lee Madden
\vskip 0.1cm
June 2021}




\begin{document}

\maketitle



%\begin{abstract}
%  \noindent Lorem ipsum dolor sit amet, consectetur adipisicing elit, sed do eiusmod tempor incididunt ut labore et dolore magna aliqua. Ut enim ad minim veniam, quis nostrud exercitation ullamco laboris nisi ut aliquip ex ea commodo consequat. Duis aute irure dolor in reprehenderit in voluptate velit esse cillum dolore eu fugiat nulla pariatur. Excepteur sint occaecat cupidatat non proident, sunt in culpa qui officia deserunt mollit anim id est laborum.
%\end{abstract}


\section{Introduction}

reference blog post and talk about the 2018 chs.

statistical questions:

Q1: how sound are statistical analyses in blog post?

Q2: can we expand that analysis to account for low-income women?

Q3: can we deep dive into vancouver?







\section{Statistical Considerations of the 2018 CHS PUMF} \label{sec:stats}


In order to protect the privacy of survey respondents (i.e.
preventing any respondent or household to be identified), data obtained
in the CHS is modified in various ways. \cite[Section~6]{chsguide} goes
into detail about the safeguards used by Statistics Canada. These include
decreasing the level of geographic detail; grouping answers into
categories in questions that contain many answers; adding random noise to
some quantitative variables; and rounding very small or large quantitative
values, which normally correspond to extreme (i.e. rare) households.
\\


The modified data, called the \textit{public use microdata file} (PUMF),
is then made public. Naturally, analyses based on the PUMF will differ from
those carried out using the full master file of Statistics Canada due
to the data modification process used to decrease disclosure risk.
\cite[Section~7]{chsguide} explains in detail all the limitations of analyses
based on the PUMF. Succinctly, however, the PUMF should not be used to carry out
statistical analyses. Rather, it should be used to conduct exploratory data
analyses that might indicate which models are appropriate and possibly
to obtain preliminary estimates of variables of interest.
\\

Related to the first statistical question, the PUMF cannot be reliably used
to measure variability, and it also does not include bootstrap weights.
Hence, practically any statistical test would produce invalid results.
This is because, even if point estimates are not necessarily way off
when compared with the values obtained from Statistics Canada's master file,
there is no way to reliably measure the quality of each estimate.
More so, the accuracy of point estimates depends on the number of variables
that are being cross-tabulated. Again, there is no objective and reliable way
to measure this accuracy.
\\

Once preliminary results have been obtained and an appropriate model
(or family of models) selected, the PUMF user guide \cite{chsguide} recommends
requesting access to the CHS master files. Statistical analyses based on
the full data files will be valid and, furthermore, can be fit
with a greater level of detail (e.g. geographically).
\\

As for the second statistical question, the PUMF contains the gender
of the reference person, i.e. the one answering the survey.
This means that accounting for gender using the PUMF would be difficult
as there is no indication of the relationship between the reference person
and the main provider of each household. It is possible that the master file
does contain information about each member of the household, including gender,
although the PUMF guide is not very clear about this.
\\

In terms of the third statistical question, the PUMF only
contains geographic information at the census metropolitan area (CMA),
which for British Columbia (B.C.) corresponds to three categories:
Vancouver, other large cities, and the rest of B.C. The confidential
master file of Statistics Canada, however, does contain more detailed
information that would allow a more granular analysis.




\section{Exploratory Data Analysis} \label{sec:eda}

here we replicate some of the plots in the blog post but separating by gender and only for bc.

mention how gender is calculated in survey and limitations.

show plots, explain.




\section{Conclusion} \label{sec:conclusion}





\clearpage
%\printbibliography
\bibliographystyle{abbrvnat}
\bibliography{ref}


\end{document}
