\documentclass[11pt]{article}
\usepackage[autonum]{macros}

% figures and bibliography
%\usepackage[sorting=nyt,backend=biber,bibstyle=alphabetic,citestyle=alphabetic]{biblatex}
%\usepackage[backend=biber]{biblatex}
\usepackage{natbib}
%\bibliography{ref.bib}


% set margins' lenghts %%%%%%
\setlength{\oddsidemargin}{0.0 in}
\setlength{\evensidemargin}{0.0 in}
\setlength{\topmargin}{-0.6 in}
\setlength{\textwidth}{6.5 in}
\setlength{\textheight}{8.5 in}
\setlength{\headsep}{0.75 in}
\setlength{\parindent}{0 in}
\setlength{\parskip}{0.1 in}

\usepackage{multirow} % for table
\usepackage[table]{xcolor} % alternating row colors in tables
\renewcommand{\abstractname}{Executive Summary} % change title of abstract

% set fancy style for margins %%%%%%
\usepackage{fancyhdr}
\pagestyle{fancy}
\fancyhf{}
\lhead{Madden, C.l}
\chead{UBC STCS}
\rhead{Diluvi, G. C.}
\cfoot{\thepage}





\title{\vspace{-2cm}\Large A Statistical View of The Right to Vancouverism: Social Reproduction Placemaking in the Revanchist City}
\author{\normalsize Gian Carlo Diluvi\footnote{Department of Statistics, University of British Columbia. Contact: \href{mailto:gian.diluvi@stat.ubc.ca}{\texttt{gian.diluvi@stat.ubc.ca}}.}}
\date{\normalsize Client: Cheryl-lee Madden
\vskip 0.1cm
June 2021}




\begin{document}

\maketitle



%\begin{abstract}
%  \noindent Lorem ipsum dolor sit amet, consectetur adipisicing elit, sed do eiusmod tempor incididunt ut labore et dolore magna aliqua. Ut enim ad minim veniam, quis nostrud exercitation ullamco laboris nisi ut aliquip ex ea commodo consequat. Duis aute irure dolor in reprehenderit in voluptate velit esse cillum dolore eu fugiat nulla pariatur. Excepteur sint occaecat cupidatat non proident, sunt in culpa qui officia deserunt mollit anim id est laborum.
%\end{abstract}


\section{Introduction}



In Section \ref{sec:stats}, we present structural equation modeling as a statistical methodology to answer the three questions of interest. In Section \ref{sec:sample}, we discuss the sampling considerations. Section \ref{sec:conclusion} concludes the report.



\section{Statistical Methodologies} \label{sec:stats}


\section{Sample Considerations} \label{sec:sample}




\section{Conclusion} \label{sec:conclusion}



%\clearpage
%\printbibliography
%\bibliographystyle{abbrvnat}
%\bibliography{ref}


\end{document}
